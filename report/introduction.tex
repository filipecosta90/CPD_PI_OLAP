% !TEX encoding = UTF-8 Unicode

\section{Introduction}
\indent

The design and development of systems that generate, collect, store, process, analyse, and query large sets of data is filled with significant challenges both hardware and software. Combined, these challenges represent a difficult landscape for software engineers.\par 
The relational database is the current solution for big data storage.
Data dependencies in databases can be seen as a binary two-way associative relation. 
Prior efforts have been made \cite{macedo2015linear} \cite{da2015benchmarking} towards a Linear Algebra (LA) approach to OLAP, in order to fully represent relational algebra in terms of linear algebra operators.
\par 

OLAP is resource-demanding and calls for efficient 	parallel approaches. 
We implemented from the start a typed linear algebra solution given special importance to the data modelling since it may limit the attainable efficiency in data-intensive systems such as OLAP, a time consuming task that accesses massive amounts. \par 
To validate the LA approach in a real case scenario and evaluate the outcome performance,the full set of TPC-H benchmarks was selected, with multiple query runs, and so far only one is reported in this work. To obtain realistic and meaningful results large datasets were considered, ranging from 1 to 32GB.\par 
For a comparative evaluation between the LA and the objected-relational database management system we selected PostgreSQL version 9.6, with roots in open source community,  to represent the relational algebra approach. \par 
Given this proximity between database relations and linear algebra, the question arises: does the linear algebra approach presents performance improvements when compared with the relational one?\par 

The following sections introduce the LA strategy (section \ref{strategy}), describe the experimental work with the sequential version of the code (section \ref{sequential}), introduce the parallelisation techniques to improve performance 	with experimental execution times \ref{parallel}, and the last section concludes with suggestions for later work.
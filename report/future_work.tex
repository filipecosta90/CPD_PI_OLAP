\section{Future Work}
\indent
\par The platform used by us for our study at Search6 is a dual-socket system equipped with two Intel\textsuperscript{\textregistered} Ivy Bridge processors. The system, referenced as compute node 652-1, has two Intel\textsuperscript{\textregistered} Xeon\textsuperscript{\textregistered} E5-2670v2 (Ivy Bridge architecture) and features 64 GB of DDR3 RAM, supported at a frequency of XXXX MHz divided in 4 memory channels.

\begin{table}[H]
\centering
  \begin{tabular}{ | L{3.5cm} | R{5cm} | }
  
    \hline
    System & compute-652-1 \\ \hline \hline
        \# CPUs & 2\\ \hline
    CPU & Intel\textsuperscript{\textregistered} Xeon\textsuperscript{\textregistered} E5-2670v2\\ \hline 
    Architecture & Ivy Bridge \\ \hline 
    \# Cores per CPU & 10 \\ \hline 
    \# Threads per CPU & 20\\ \hline 
    Clock Freq. & 2.5 GHz\\ \hline \hline 
    L1 Cache & 320KB \newline 32KB per core\\ \hline 
    L2 Cache & 2560KB  \newline  256KB per core \newline\\ \hline 
    L3 Cache & 25600KB \newline shared \\ \hline \hline 
    Inst. Set Ext. & SSE4.2 \& AVX \\ \hline 
        \#Memory Channels & 4\\ \hline \hline

    Vendors Announced Peak Memory BW & 59.7 GB/s\\ \hline
    Measured\footnote{Stream Benchamrk} Peak Memory BW & 58.5GB/s\\ \hline
  \end{tabular}
     \caption{Architectural characteristics of the two evaluation platforms.}
     \label{table:characterization}
\end{table}


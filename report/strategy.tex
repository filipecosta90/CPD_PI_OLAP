\section{Linear Algebra Strategy}
\indent
\subsection{Towards a linear algebra semantics for SQL}

Inspired by point-free relational data processing, an alternative roadmap for parallel online analytical processing (OLAP) can be achieved based on encoding data in matrix format and relying thereupon solely on LA operations\cite{macedo2011middle}. 

\subsubsection{Encoding data in matrix format}

As example of raw data consider the displayed table \ref{table:example_table} where each row records the order key, quantity, return flag, line status and ship date from an given TPC-H benchmark lineitem table. In order to facilitate data association the column number in the corresponding lineitem table was included above each data column.
% !TEX encoding = UTF-8 Unicode

\begin{table}[H]


\caption{ Collection of raw data (adapted from TPC-H benchmark lineitem table). }
\label{table:example_table}
\scriptsize
\centering
\begin{tabular}{ |  L{1.5cm} |  L{1.5cm}  |  L{1.5cm}  |  L{1.5cm}  |  L{1.5cm} |   } 
\hline
\#1	&	\#5	&	\#9	&	\#10	&	\#11	  \\ 
l\_orderkey	&	l\_quantity	&	l\_returnflag	&	l\_linestatus	&	 l\_shipdate 	  \\ \hline
\hline
1	&	17	&	N	&	O	&	1996-03-13	  \\ \hline
1	&	36	&	N	&	O	&	1996-04-12	  \\ \hline
1	&	8	&	N	&	O	&	1996-01-29	  \\ \hline
1	&	28	&	N	&	O	&	1996-04-21	  \\ \hline
1	&	24	&	N	&	O	&	1996-03-30	  \\ \hline
1	&	32	&	N	&	O	&	1996-01-30	  \\ \hline
2	&	38	&	N	&	O	&	1997-01-28	  \\ \hline
3	&	45	&	R	&	F	&	1994-02-02	  \\ \hline
3	&	49	&	R	&	F	&	1993-11-09	  \\ \hline
3	&	27	&	A	&	F	&	1994-01-16	  \\ \hline
\end{tabular}
\end{table}


To obtain useful information from raw data, which in OLAP systems escalated to Terabytes of information, we need to summarise the data by selecting attributes of interest and exhibiting their inter-relationships.\par 
Consider the following simplified TPC-H query 1: "How many items were sold per return flag and line status?". For this particular question, the necessary attributes to answer the query are present on table lineitem, being \textbf{return flag}, \textbf{line status}, and \textbf{quantity}. In relational algebra that question could be easily answered by the following SQL code:

\lstinputlisting[style]{sql/simple_query_1.sql} %input de um ficheiro


which would produce the result presented on table \ref{table:results_simple_query_1}
% !TEX encoding = UTF-8 Unicode
\begin{table}[H]
\caption{Simplified query-1 result from the collection of raw data (adapted from TPC-H benchmark lineitem table). }
\label{table:results_simple_query_1}
\scriptsize
\centering
\begin{tabular}{ |  L{1.5cm} |  L{1.5cm}  |  L{1.5cm}  |    } 
\hline
\#9	&	\#10		 & \#5	  \\ 
l\_returnflag	&	l\_linestatus	&	 l\_quantity 	  \\ \hline
\hline
N	&	O	&	183	  \\ \hline
R	&	F	&	94	  \\ \hline
A	&	F	&	27	  \\ \hline
\end{tabular}

\end{table}


Aggregations like the presented on the prior simplified query occur in all TPC-H queries, hence performance of group-by and aggregation is quite important. That matter will be addressed in the later sections of the report.\par 
Regarding expressing the OLAP  in therms of LA, the key resides in expressing operations in the form of matrix algebra expressions.  In this particular example, we should be able to build three matrices, one for each attribute, with each matrix being correlated to relational algebra as the row storage of columns \#5, \#9, and \#10. However, in order to do so, we need to find a two-way association between the presented string on the rows \#9 and \#10 and an unique integer identifier. Our proposed solution encodes strings recurring to Gnome \textbf{Gquarks} \cite{gquarks} - a two-way association between a non-zero unsigned int and a char * - based on a thread safe hashtable. \par 
By order of appearance, each unique string will be associated to an unique unsigned integer. Repeated strings will be associated to the prior corresponding unsigned integer. The resulting unsigned integer value range will start in the number 1. The value 0 in GQuarks is associated to NULL. Since in the proposed solution row and column numbers will respect  C-style arrays notation both column and row numbering will start on 0. The GQuark unsigned integer value decremented by 1  will represent the row position on the matrix, and the register number, starting at 0, will represent the column position on the matrix. \par 
We can now verify that there will be only one element per column of the matrix in order to maintain the two-way association between an register and its corresponding string. There is also the possibility of direct associating an register number with a value, wether integer or floating point. The referred matrices will be diagonal matrices, in which the element value is directly associated with the register.\par 
Two types of matrices have now been presented:
\begin{enumerate}
\item Projection Matrices - which recur to Gnome Gquarks, in which the Gquark decremented by 1, represents the row position of the matrix, and the register number, starting at 0, will represent the column position on the matrix.
\item Measure Matrices -  direct associating an register number with a value, wether integer or floating point,  which the element value is directly associated with the register number, and consequently the column and row position.
\end{enumerate}
 
We can now build the necessary matrices for the given examples. The two-way association between char* and unsigned integer is also presented in table \ref{table:association_quarks_row} in order to aid the example comprehension. 
% !TEX encoding = UTF-8 Unicode
\begin{table}[H]
\caption{Two-Way association between GQuarks and Row Number, for rows \#9 and \#10 presented in the collection of raw data (adapted from TPC-H benchmark lineitem table). }
\label{table:association_quarks_row}
\scriptsize
\centering
\begin{tabular}{ |  L{1.5cm} |  L{1.5cm}  |  L{1.5cm}  |    } 
\hline
String	&	GQuark	&	Row \#	  \\ \hline
\hline
N	&	1	&	0	  \\ \hline
R	&	2	&	1	  \\ \hline
A	&	3	&	2	  \\ \hline
O	&	4	&	3	  \\ \hline
F	&	5	&	4	  \\ \hline
\end{tabular}

\end{table}







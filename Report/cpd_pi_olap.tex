% !TEX encoding = UTF-8 Unicode

%%%%%%%%%%%%%%%%%%%%%%%%%%%%%%%%%%%%%%%%%
% Journal Article
% LaTeX Template
% Version 1.3 (9/9/13)
%
% This template has been downloaded from:
% http://www.LaTeXTemplates.com
%
% Original author:
% Frits Wenneker (http://www.howtotex.com)
%
% License:
% CC BY-NC-SA 3.0 (http://creativecommons.org/licenses/by-nc-sa/3.0/)
%
%%%%%%%%%%%%%%%%%%%%%%%%%%%%%%%%%%%%%%%%%

%----------------------------------------------------------------------------------------
%	PACKAGES AND OTHER DOCUMENT CONFIGURATIONS
%----------------------------------------------------------------------------------------

\documentclass[twoside]{article}

\usepackage{lipsum}

\usepackage[sc]{mathpazo} % Use the Palatino font
\usepackage[utf8]{inputenc}
\usepackage[english]{babel}
\usepackage[T1]{fontenc} % Use 8-bit encoding that has 256 glyphs
%\linespread{1.05} % Line spacing - Palatino needs more space between lines
\usepackage{microtype} % Slightly tweak font spacing for aesthetics

%\usepackage[hmarginratio=1:1,top=32mm,columnsep=20pt]{geometry} % Document margins
\usepackage[top=20mm,left=8mm,right=8mm,bottom=20mm]{geometry}

\usepackage{multicol} % Used for the two-column layout of the document
\usepackage[hang, small,labelfont=bf,up,textfont=it,up]{caption} % Custom captions under/above floats in tables or figures
\usepackage{booktabs} % Horizontal rules in tables
\usepackage{float} % Required for tables and figures in the multi-column environment - they need to be placed in specific locations with the [H] (e.g. \begin{table}[H])
\usepackage{hyperref} % For hyperlinks in the PDF

\usepackage{lettrine} % The lettrine is the first enlarged letter at the beginning of the text
\usepackage{paralist} % Used for the compactitem environment which makes bullet points with less space between them

\usepackage{abstract} % Allows abstract customization
\renewcommand{\abstractnamefont}{\normalfont\bfseries} % Set the "Abstract" text to bold
\renewcommand{\abstracttextfont}{\normalfont\small\itshape} % Set the abstract itself to small italic text

%2º espacamento antes da section
\usepackage{titlesec} % Allows customization of titles
%\titlespacing*{\section}{0pt}{0pt}{0pt}
\renewcommand\thesection{\Roman{section}} % Roman numerals for the sections
\renewcommand\thesubsection{\Roman{subsection}} % Roman numerals for subsections
\titleformat{\section}[block]{\large\scshape\centering}{\thesection.}{1em}{} % Change the look of the section titles
\titleformat{\subsection}[block]{\large}{\thesubsection.}{1em}{} % Change the look of the section titles

\usepackage{fancyhdr} % Headers and footers
\pagestyle{fancy} % All pages have headers and footers
\fancyhead{} % Blank out the default header
\fancyfoot{} % Blank out the default footer
\fancyhead[C]{University of Minho $\bullet$ CPD Integrated Project 2015-2016} % Custom header text
\fancyfoot[RO,LE]{\thepage} % Custom footer text


%%% ADDED BY ME %%%%%%
%\floatstyle{boxed} 
\restylefloat{figure}
\usepackage{graphicx}
\usepackage{caption}
\usepackage{subcaption}
\usepackage{amsfonts}
\usepackage{listings,mdframed}
\usepackage{fancyvrb}
\usepackage{cleveref}
\usepackage{mathtools}
\usepackage{amsmath}
\usepackage{color}
\usepackage{relsize}
\usepackage{array}
\newcolumntype{L}[1]{>{\raggedright\let\newline\\\arraybackslash\hspace{0pt}}m{#1}}
\newcolumntype{C}[1]{>{\centering\let\newline\\\arraybackslash\hspace{0pt}}m{#1}}
\newcolumntype{R}[1]{>{\raggedleft\let\newline\\\arraybackslash\hspace{0pt}}m{#1}}
\usepackage{tabularx,caption}


%%% For using norm || %%%%%
\newcommand{\norm}[1]{\left\lVert#1\right\rVert}


\usepackage{xcolor}
\lstdefinestyle{BashInputStyle}{
  language=bash,
  basicstyle=\small\sffamily,
  numbers=left,
  numberstyle=\tiny,
  numbersep=3pt,
  frame=tb,
  columns=fullflexible,
  backgroundcolor=\color{yellow!20},
  linewidth=0.9\linewidth,
  xleftmargin=0.1\linewidth
  %moredelim=**[is][\color{red}]{@}{@},
}


\lstdefinestyle{colorcode}{
  language=C,
  basicstyle=\small\sffamily,
  numberstyle=\tiny,
  numbersep=6pt,
  emptylines=1,
  breaklines=true,
  columns=fullflexible,
  linewidth=0.9\linewidth,
  xleftmargin=0.1\linewidth
  basicstyle=\ttfamily\color{black},
  moredelim=**[is][\color{red}]{@}{@},
  moredelim=**[is][\color{orange}]{|}{|}
}

\newcommand*{\Package}[1]{\texttt{#1}}

\lstset{
    numbers=left,
	language=C,
	keywordstyle=\bfseries\ttfamily\color[rgb]{0,0,1},
	identifierstyle=\ttfamily,
	commentstyle=\color[rgb]{0.133,0.545,0.133},
	stringstyle=\ttfamily\color[rgb]{0.627,0.126,0.941},
	showstringspaces=false,
	basicstyle=\small,
numberstyle=\tiny,
numbers=right,
	stepnumber=1,
	numbersep=10pt,
	tabsize=1,
	breaklines=true,
	prebreak = \raisebox{0ex}[0ex][0ex]{\ensuremath{\hookleftarrow}},
	breakatwhitespace=false,
	aboveskip={1.5\baselineskip},
  columns=fixed,
  upquote=true,
  extendedchars=true,
 frame=single,
 inputencoding=utf8,
    literate={á}{{\'a}}1 {ã}{{\~a}}1 {â}{{\~a}}1 {é}{{\'e}}1 {ê}{{\'e}}1 {ç}{{\'c}}1 {ú}{{\'u}}1 {ó}{{\'o}}1 {í}{{\'i}}1,
 %backgroundcolor=\color{lbcolor},
}

%----------------------------------------------------------------------------------------
%	TITLE SECTION
%----------------------------------------------------------------------------------------

\title{\vspace{-15mm}\fontsize{24pt}{10pt}\selectfont\textbf{Optimisation of a Linear Algebra Approach to OLAP}} % Article title

\author{
\large
\textsc{Filipe Oliveira} - \textsc{A57816}\\
\normalsize \href{mailto:a57816@alunos.uminho.pt}{a57816@alunos.uminho.pt}
\vspace{-5mm}
\and
\textsc{Sérgio Caldas} - \textsc{A57779}\\
\normalsize \href{mailto:a57779@alunos.uminho.pt}{a57779@alunos.uminho.pt}
}



%----------------------------------------------------------------------------------------

\begin{document}

\maketitle % Insert title

\thispagestyle{fancy} % All pages have headers and footers

%----------------------------------------------------------------------------------------
%	ABSTRACT
%----------------------------------------------------------------------------------------

\begin{abstract}
\indent % Dummy abstract text
\par Online Analytical processing (OLAP) systems, perform multidimensional analysis of business data and provides the capability for complex calculations, trend analysis, and sophisticated data modeling. 
All the referred analysis depends on Relational Algebra, which lack algebraic properties, and qualitative and quantitative proofs for all the relational operator.
The proposed solution focus on a typed linear algebra approach, encoding OLAP functionality solely in terms of Linear Algebra operations (matrices).
\par It has been argued that linear algebra (LA) is better suited than standard relational algebra for formalizing and implementing queries in on-line multidimensional data analysis \cite{macedo2015linear} \cite{da2015benchmarking}. This can be achieved over a small LA sparse matrix kernel which, further to multiplication and transposition, offers the Kronecker, Khatri-Rao and Hadamard products.

\end{abstract}
\vspace{0.5cm}

%----------------------------------------------------------------------------------------
%	ARTICLE CONTENTS
%----------------------------------------------------------------------------------------
\begin{multicols}{2} % Two-column layout throughout the main article text
\section{Introduction}
\indent
\par In a initial fase we implemented from the start a typed linear algebra solution, and evaluate its performance in a real word scenario, using different datasets. The datasets was produced with TPC-H Benchmark, was produced multiples datasets with different data sizes (1, 2, 4, 8, 16, 32 Gb) in order to obtain a realistic results.

\section{Hardware Characterization}
\indent
\par The platform used by us for our study at Search6 is a dual-socket system equipped with two Intel\textsuperscript{\textregistered} Ivy Bridge processors. The system, referenced as compute node 652-1, has two Intel\textsuperscript{\textregistered} Xeon\textsuperscript{\textregistered} E5-2670v2 (Ivy Bridge architecture) and features 64 GB of DDR3 RAM, supported at a frequency of XXXX MHz divided in 4 memory channels.

\begin{table}[H]
\centering
  \begin{tabular}{ | L{3.5cm} | R{5cm} | }
  
    \hline
    System & compute-652-1 \\ \hline \hline
        \# CPUs & 2\\ \hline
    CPU & Intel\textsuperscript{\textregistered} Xeon\textsuperscript{\textregistered} E5-2670v2\\ \hline 
    Architecture & Ivy Bridge \\ \hline 
    \# Cores per CPU & 10 \\ \hline 
    \# Threads per CPU & 20\\ \hline 
    Clock Freq. & 2.5 GHz\\ \hline \hline 
    L1 Cache & 320KB \newline 32KB per core\\ \hline 
    L2 Cache & 2560KB  \newline  256KB per core \newline\\ \hline 
    L3 Cache & 25600KB \newline shared \\ \hline \hline 
    Inst. Set Ext. & SSE4.2 \& AVX \\ \hline 
        \#Memory Channels & 4\\ \hline \hline

    Vendors Announced Peak Memory BW & 59.7 GB/s\\ \hline
    Measured\footnote{Stream Benchamrk} Peak Memory BW & 58.5GB/s\\ \hline
  \end{tabular}
     \caption{Architectural characteristics of the two evaluation platforms.}
     \label{table:characterization}
\end{table}


%---------------------------------------------------------------------------------------
%	REFERENCE LIST
%--------------------------------------------------------------------------------------
\bibliographystyle{plain}
\bibliography{references.bib}

%---------------------------------------------------------------------------------------
\end{multicols}
\end{document}



%\par In order to compute algorithm complexity correctly, we need to know the values for \textcolor[rgb]{1,0.5,0}{\textbf{md->movemx}}, and \textcolor[rgb]{1,0.5,0}{\textbf{md->mdsize}}. So, after inspecting \textbf{MD.c} file we can see that \textcolor[rgb]{1,0.5,0}{\textbf{md->movemx}} is 50. 
%The \textcolor[rgb]{1,0.5,0}{\textbf{md->mdsize}} value depends on the value of \textbf{md->mm} value. This values depends on datasizes[md->size] wich is on main.c file and is a parameter of \textbf{initialiseMD} function. Since  the value of \textbf{md->mm} doesn't change the complexity of the algorithmm we will use de value on datasizes[1] wich is \textbf{13}.


%\par So, to compute \textcolor[rgb]{1,0.5,0}{\textbf{md->mdsize}} we'll need to perform the following calculations:


%\begin{verbatim}
%    md->mdsize = md->mm * md->mm * md->mm * 4.
%    md->mm = datasizes[1]
%    datasizes[1] = 13
%    md->mm = 13
%\end{verbatim}

%\textbf{md->mdsize} is 13 * 13 * 13 * 4 wich is equal to 8788. Thats the total executions of \textbf{force} function we'll consider. Inside force, we'll also need to contabilize the complexity of \textbf{calculePow}, \textbf{third\_newton\_law}, \textbf{update\_static\_variables} and \textbf{updateForce}.
%All of these functions just perform memory access and has no loop so we'll consider they execute $50 \times n\times n$ times. For those functions we achieved the complexity $\mathcal{O}(n^{2})$. For \textbf{force} we can achieve the following formulation to determine the complexity:


